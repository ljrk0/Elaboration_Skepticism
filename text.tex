\documentclass[12pt,a4paper]{article}
\usepackage[utf8]{inputenc}
\usepackage[ngerman]{babel}
\usepackage[T1]{fontenc}
\usepackage{csquotes}

%\usepackage{graphicx}
%\linespread{1.5}
\usepackage[left=4cm,right=2cm,top=3cm,bottom=3cm]{geometry}
%\usepackage[left=2cm,right=8cm,top=3cm,bottom=3cm]{geometry}
\usepackage[modulo, pagewise]{lineno}
\author{Leonard König}
\title{Skeptizismus - Flucht der Philosophie vor der Frage nach der Erkenntnis?}
% personalpronomen? Ja, Nein?

\setlength{\parskip}{1em}
\usepackage[
	natbib=true,
    backend=biber,
    style=numeric,
    citestyle=numeric,
    sorting=nty,
    sortlocale=de_DE,
]{biblatex}
\addbibresource{text.bib}

\begin{document}
\maketitle
\tableofcontents
\newpage
%
\section{Hinführung zum Thema: Sokrates}
\linenumbers
\glqq Ich weiß, dass ich nichts weiß\grqq. Es gibt verschiedene Auslegungen dieses wohl berühmtesten Zitats Sokrates'.% Sokrates'?
\\Einerseits kann man es als eine Devise des sokratischen, methodischen Zweifels auffassen - also der immerwährenden Skepsis, um zur Erkenntnis zu gelangen - , andererseits kann man das Zitat durchaus anders interpretieren: Egal was ich tue, ich kann nie zu einer Erkenntnis kommen, was als Interpretationsansatz mein erster Berührungspunkt mit der philosophischen Haltung des Skeptizismus war.\\
Mir erschien dieser Schluss immer logisch: Woher soll ich \glqq wissen\grqq? Was ist überhaupt \glqq Wissen\grqq\ oder gar \glqq Erkenntnis\grqq\ ? Ist \glqq Erkenntnis\grqq\ überhaupt möglich?\\
Doch zunächst werde ich diese Fragen zurückstellen und stattdessen mich einigen Definitionen widmen.
	\subsection{Definitionen}
		\subsubsection{Skeptizismus}
Der Duden sagt über den Skeptizismus:\\
\glqq Im weiteren Sinn philosophische Positionen, die Wahrheitsansprüchen gegenüber Verzicht und Zurückhaltung üben und sorgfältige kritische Prüfung verlangen.\\
Im strengeren Sinne Richtungen, die die Beweisbarkeit von Wahrheit (entweder prinzipiell oder partiell) in Zweifel ziehen.\grqq \\
Einer der moderneren Hauptvertreter des Skeptizismus ist der Philosoph Hume.%Quelle % Zitat checken, cite  nutzen etc. %unter anderen?
		\subsubsection{Agnostizismus}
Der Begriff \glqq Agnostizismus\grqq\ wurde vom Biologen Thomas Henry Huxley geprägt. Agnostizismus ist oft als eine Unterposition des Skeptizismus aufzufassen, in der man davon ausgeht, dass über jegliche oder nur bestimmte überirdische% irden/irdlich/mystisch?
\ Ideen - wie beispielsweise einem Gott oder der Wiedergeburt - keine Erkenntnis zu gewinnen ist. In der Regel fasst man den Begriff noch enger um eine Religion, sprich als Abgrenzung zu The- und Atheisten.\\ 
Vereinzelt jedoch wird der Begriff Agnostizismus synonym zum Skeptizismus verwendet. Dabei geht man davon aus, dass, etymologisch gesehen, Agnostizismus \glqq ohne Wissen\grqq\ bedeutet.
		\subsubsection{Solipsismus}
Der Solipsismus ist eine philosophische Schule, die Allem außer meiner Selbst gegenüber eine skeptische Haltung einnimmt. Solche Philosophen vertreten die Meinung, dass entweder nur ihre eigene Existenz gesichert ist, oder, dass die Bedeutung der Wahrnehmung vom Zustand des eigenen (denkenden) Ichs abhängt \cite{iep_solipsis}.%erster Satz Grammatik
\section{Skeptizismus im historischen Kontext}
	\subsection{Antike - die Sophisten} %TODO Ordnen, historisch u. nach Position
Ein kurzer Überblick: Die Sophisten waren Wanderlehrer, die Diskussion und antike Formen der Philosophie lehrten. Hierbei interessiert natürlich vor allem die Epistemologie\slash Erkenntnistheorie der Antike. Sie haben jedoch logischerweise nicht unbedingt ähnliche Ansichten, was diese Richtung der Philosophie betrifft, schließlich waren die Schulen teilweise um Jahrhunderte entfernt. Während Protagoras den Menschen für das Maß aller Dinge hält \cite{enc_brit_sophist}% Begründung logischerweise <- fixed?
, ist Gorgias einem dogmatischen Skeptizismus zuzuordnen:
\begin{itemize}
\item Nichts existiert;
\item Wenn etwas existiert, dann kann man nichts darüber wissen; und
\item Wenn man etwas darüber wissen kann, dann kann dieses nicht mitgeteilt werden. \cite{iep_on-the-nonex} 
%\item Wenn es mitgeteilt werden kann, kann es nicht verstanden werden.
\end{itemize}
Radikale Skeptiker würden selbst der letzten Aussage noch skeptisch gegenüber stehen.\\
Pyrrho beispielsweise meinte: \glqq unsere Sinneserfahrungen und Ansichten sind weder wahr, noch falsch\grqq\ \cite{greek_stough}.
Es gibt jedoch einige Strömungen des Skeptizismus. So gibt es den akademischen Skeptizismus, der die Aussage trifft, dass Wissen nicht möglich ist. Der Skeptizismus nach Pyrrho geht nicht so dogmatisch vor, sondern behauptet, man könne über die Aussage selbst, also, dass kein Wissen möglich ist, keine wirkliche Aussage treffen kann (radikaler Skeptizismus). % %TODO http://en.wikipedia.org/wiki/Philosophical_skepticism
Eigentlich verfolgte Pyrrho jedoch das Ziel der Ataraxia, der mentalen Unerschütterlichkeit.
% ( %TODO http://en.wikipedia.org/wiki/Pyrrho ; Ataraxia <=> Lethargie)

Der siebte Leiter der antiken Akademie, Arkesilaos, begründete den akademischen Skeptizismus. Er schloss sich dabei seinen Vorgängern an, dass es keine Wahrheit gäbe, weil zu beliebigen (philosophischen) Aussagen gleichberechtigte Gegenargumente oder Positionen zu finden seien.
	\subsection{Mittelalter - akademischer Skeptizismus versus Kirche}
%TODO Bis hier: 5min
Diese Richtung des Skeptizismus dauerte noch einige Zeit an und prägte sogar den Kirchenlehrer Augustinus. Schließlich wurde sein Interesse an der Philosophie durch Ciceros Werke geweckt, einem Anhänger des akademischen Skeptizismus im ersten Jahrhundert vor Christus. Er selber vertrat sogar diese Philosophie und hat sie benutzt, um gegen den Manichäismus zu argumentieren, einer Offenbarungsreligion, der er vor seiner skeptischen Zeit selber angehörte.\\% evtl umformulieren
Später jedoch wurde er zu einem der größten Kritiker seiner ehemaligen Schule. Zwar bezog er sich - wie auch die Akademiker - auf Platon, jedoch legt er die Bibel im Sinne der platonischen Philosophie aus, was in einer sehr religiösen Richtung mündete.\\% , nachdem er den \glqq Römerbrief\grqq\ des Paulus gelesen hatte.\\
Dennoch beschäftigte sich Augustinus weiterhin mit der Wahrheit und nahm dabei Descartes \glqq cogito ergo sum\grqq\ voraus:\\
\glqq Wird jemand darüber zweifeln, dass er lebt, sich erinnert, Einsichten hat, will, denkt, weiß und urteilt? [\ldots] Mag einer auch sonst zweifeln, über was er will, über diese Zweifel selbst kann er nicht zweifeln\grqq\ \cite{de_trini_x}.
%TODO Zitat hervorheben, evtl. Manichäismus raus; dafuer mehr Popper
	\subsection{Skeptizismus von Descartes bis Kant}
		\paragraph{Descartes, Locke, Hume, Kant}
Descartes hat zwei sehr bedeutende Dinge von seinen Vorgängern übernommen: Den methodischen Zweifel (von Sokrates) und, wie gerade erwähnt, den Ausspruch \glqq ich zweifle, also bin ich\grqq.\\% -- "(eventuell unwissend)" ?
Descartes hat damit versucht, ohne jede Voraussetzung von einer Wahrheit zumindest \emph{irgendetwas} zu \glqq beweisen\grqq, etwas was über allem Zweifel erhaben ist.\\
Er meinte, man könne zweifeln so viel man möge, an allem und jedem, aber, genau das zeige, man zweifele. Also existiere man.\\ % -- möge ++wolle ?
\emph{Doch: Weiß man, dass man wirklich zweifelt?} % evtl. als Überleitung nutzen, sonst verlaengern. Einbetten: "Ich bezweifle das."

Gegner des Rationalisten Descartes waren die Empiristen. Empiriker meinen, Erkenntnis könne (nur) durch Erfahrung gewonnen werden - im Gegensatz zum Rationalismus, der lehrt man habe angeborene Ideen, die nur bestätigt werden müssten. Deshalb ist der Empirismus oft auch \glqq näher\grqq\  Skeptizismus, schließlich liegt es nahe, dass, wenn man nur durch Wahrnehmung zu Erkenntnis gelangt, diese Wahrnehmung verfälscht ist und man damit zu keiner \glqq echten\grqq\ Erkenntnis kommt. David Hume, ein britischer Empirist, war genau dieser Auffassung.\\ % naheligend doppelt; dass entfernen
Diese Probleme von den Arten der Erkenntnis und ob es überhaupt eine Erkenntnis gibt, sowie über die Problematik der Induktion behandelt er in seinem ersten großen Werk \glqq A Treatise of Human Nature\grqq.

Fragen, die auch Kant beschäftigt haben. Er selber schrieb Hume seine \glqq Befreiung\grqq\ vom Dogmatismus zu. Kant gab zu, dass Humes skeptische Folgerungen% erwaehnen von Folgerungen
\ zwar richtig seien, jedoch solle man es dabei nicht belassen, sondern, so meint Kant, zu einer  \glqq gereiften [\ldots] Urteilskraft\grqq\ gehöre es auch, dass diese \glqq feste und ihrer Allgemeinheit nach bewährte Maximen zum Grunde hat\grqq\ \cite{kritik}.

Was Kant damit meinte: Der Skeptizismus sei zwar ein wichtiges Werkzeug, da man mit ihm das eigene Leben skeptisch hinterfragen und reflektieren könne; man solle aber den Skeptizismus nicht als philosophische Position ansehen, mit der man die eigene Lethargie rechtfertige, indem man resigniert schließe: \glqq Ich kann ja sowieso nichts wissen\grqq .
%TODO evtl. nach dem Motto; Kant mehr betonen; Zitat evtl. hier
% evtl. besser formulieren.
%
	\subsection{Moderner Skeptizismus}
%TODO überarbeiten; holprig; ~10min; Kritizismus erlaeutern
Mit dieser Argumentation betritt man jedoch ethisches Terrain, denn: Warum sollte ich nicht meine nihilistische Lethargie \textbf{nicht} über den Skeptizismus begründen dürfen? Ist es möglich die Lethargie nicht nur zu begründen, sondern auch zu rechtfertigen? Schließlich: Ich kann auch nicht wissen, ob es nicht \emph{doch} eine erkennbare Welt gibt. Schuldete ich nicht dieser meinen Beitrag, meine Mithilfe?\\
An diesem Punkt setzt Nietzsche an. Er hält eine solche Argumentation für logisch valide, moralisch jedoch für verwerflich.\\% überleitung zu nietzsche und nelson -> ethik / politologie bzw. gesellschaft
Der Nihilismus bedeutet für Nietzsche, die obersten Werte der Moral zu entwerten, da durch den Verlust der Metaphysik kein Fundament der Moral mehr gegeben ist. Aus diesem Grund versucht er mit der \glqq Umwertung aller Werte\grqq\ eine neue Basis für die Ethik zu schaffen, in deren Rahmen er den Nihilismus verurteilt. Auf diese ethische Argumentation, kann jedoch im Verlauf dieses Vortrages nicht weiter eingegangen werden, da dies den Zeitrahmen sprengen würde.\\
Ein weiterer Vertreter der Position, dass Ethik und Skeptizismus mit einander verbunden sind, ist Leonard Nelson.\\
Wie die Pyrrhoniker, versuchte er in seinem Hauptwerk \glqq Die Unmöglichkeit der Erkenntnistheorie\grqq\ , Selbiges zu beweisen.
% berguendung der ausführlichkeit der behandlung von leonard nelson

% Überleitung
% Popper:
In seiner Wissenschaftstheorie kritisiert Popper den Einsatz der Induktion als wissenschaftliches Mittel zum Erkenntnisgewinn. Ursprünglich vertritt er damit also eine skeptizistische Position, versucht diese jedoch dann zu überwinden.\\
Er spricht sich dafür aus, dass auf anderem Wege gesicherte Erkenntnisse gewonnen werden können.\\
Diese Position nennt man kritischen Realismus.% Rationalismus? sicher? uebepruefen!
%
%
		% Popper, Nietzsche, Schopenhauer
		% Stichwort: Neukantianismus
		% Nelson; Kritizsimus
\section{Reflexion - Skeptizismus als Lebenseinstellung?}
Wir haben jetzt also eine ganze Liste an Fragen abzuarbeiten:
\begin{itemize}
\item Wie ist Skeptizismus in diesem Kontext der Fragestellung zu verstehen?\\
Eventuell sogar: Sind verschiedene Haltungen des Skeptizismus hier differenziert zu betrachten?
\item Ist der Skeptizismus in sich logisch konsistent?
\item Hat der Skeptizismus einen epistemologischen Sinn?\\
Falls nein, ist er eventuell trotzdem sinnvoll in einem abstrakteren Kontext? %konkreter
% ende der reflexion: aber eine frage habe ich vergessen:
% der skeptizismus ist erst eine flucht, wenn man ihn benutzt, sonst bleibt er eine moeglichkeit
% Und da "die Philosophie" nicht "fluechten" kann, ist es eben nur letzteres, maximal
\end{itemize}
%
Wie schon erwähnt, gibt es verschiedene \glqq Arten\grqq\  des Skeptizismus. Um die Leitfrage zu beantworten, macht es zwar Sinn, eine grobe Übersicht über die einzelnen philosophischen Standpunkte zu gewinnen, jedoch ist für die Leitfrage vor Allem der Skeptizismus in seinen erkenntnistheoretischen und metaphysischen Aspekten relevant; das heißt:\\
Der Fokus liegt auf dem sehr radikalen Skeptizismus von Pyrrho aus der Antike und den daraus hervorgehenden Positionen Humes, Kants und schließlich die des Kritizismus.
%TODO "Der Fokus" -- unser Fokus!

Einleitend kann man sagen, dass der epistemologische Skeptizismus ein durchaus abstraktes Thema mit sehr \glqq einfachen\grqq\ Argumenten in sich konsistent zu behandeln versucht. Wenn man dies jedoch schon als \glqq Flucht\grqq\ bezeichnet, hat man vorschnell geurteilt.

Der Skeptizismus ist ein Versuch der Philosophie eine Erkenntnistheorie nur auf Basis der Logik aufzubauen. Kein echter Skeptiker würde eine Erkenntnistheorie, die ebenso logisch und ohne Axiome auskommt, ablehnen, selbst wenn sie mehr Aussagen erlaubte. Schon aus diesem Grund hat der Skeptizismus zumindest eine Daseinsberechtigung - wir wollen aber noch mehr:\\
Wir wollen wissen, ob er auch noch heute eine Sinnhaftigkeit darüber hinaus hat, und eben nicht nur eine \glqq Flucht\grqq\ ist.

Da der Skeptizismus keine Aussage über unsere Erkenntnis liefert, genauer, er liefert die Aussage, dass er keine Aussage darüber geben kann, ist er in diesem Sinne \textbf{nicht} von erkenntniserweiterndem Wert, abseits eben dieser (auch anzweifelbaren) Erkenntnis.\\
Jedoch soll dies nicht die einzige Beurteilung einer philosophischen Richtung sein. 

Es besteht kein Zweifel darüber, dass der Skeptizismus durchaus eine Berechtigung der Lethargie darstellen kann. Er bietet damit die Grundlage einer \glqq Fluchtmöglichkeit\grqq\ . Schließlich ist er ein Mittel, um mit reiner Logik zu jeder Antwort einen Zweifel zu finden. Dadurch ermöglicht er es aber auch einem, diesen Zweifel als Sokrates'schen Zweifel zu nutzen - wie es auch Descartes in seiner Hinführung zum \glqq cogito ergo sum\grqq\ tat. Nur brach er - bei seinem \glqq letzten Zweifel\grqq , dem Zweifel an dem eigenen Denken - ab. Hätte es Sinn gemacht diesen Zweifel weiterzuführen? Das hängt wohl eindeutig von der Intention ab. Beabsichtigt man wirkliche, weitere Erkenntnis zu gewinnen, so ist wohl eine weitere Fortsetzung schwerlich behilflich, weil man eben gewonnene Erkenntnis wieder eintauscht, um weiteren Zweifel zu erhalten.\\
Ist man jedoch auf der Suche nach Abstraktion und Phantasie ist es interessant, diesen Weg der Begegnung mit der philosophischen Hydra zu beschreiten. Meint man eine Antwort gefunden zu haben, so eröffnen sich dem Zweifler gleich mehrfache Wege fortzuschreiten und die Ewigkeit des Zweifels zu erkunden.\\
Es geht dem Skeptiker nicht darum, Beweise führen zu können, es geht ihm darum, den Zweifel zu erkunden, waghalsig Thesen aufzustellen und zu verwerfen.

% evtl. zu viel...
Noch interessanter wird es jedoch, wenn wir uns überlegen, \emph{was} alles ist was wir eventuell auch nur scheinbar denken. Denn irgendetwas wissen wir ja doch, oder tun wir - nur wissen wir nicht, ob wir dieses Etwas auch \glqq Wissen\grqq , respektive \glqq Tun\grqq , nennen können.\\
Was ist diese \glqq Überlegung\grqq , dass wir eventuell keine Erkenntnis haben können, wenn nicht Erkenntnis? Informationen?\\
Letztendlich gibt es also diese binären Möglichkeiten: Wir denken -oder nicht.
Falls wir denken, \glqq existiert\grqq\ dieses Denken. Falls nicht, was ist es dann? Es wird \glqq uns\grqq\ vorgegaukelt, wir dächten. Ergo \glqq denken\grqq\  wir nicht, vielmehr \glqq wird für uns gedacht\grqq . 
%TODO evtl. "Programm" klairifizieren: Steuerung; "Abschied vom Subjekt"; "Makrokosmos"; Variierung von "programmiert"
Beispielsweise ist es vorstellbar, dass wir eigentlich nur programmierte Entitäten sind - programmiert, sodass sie die eigene Trivialität nicht erkennen und sich als \glqq viel zu komplex\grqq\ einstufen, um programmiert zu sein. Diese Entitäten \glqq denken\grqq\ also nur Prozesse, die eigentlich lediglich durch das Programm vorgegeben sind. Von unabhängigen Denken können wir nicht sprechen, aber eher vom Ablaufen eines Algorithmus'.\\ Andererseits, damit diese Entitäten zuerst einmal \glqq denken zu denken\grqq , muss eben dieses Übergeordnete existieren, ein Programm, in dem das \glqq Denken\grqq\ festgeschrieben ist. Wir kommen also zum Schluss, unsere Gedanken \glqq existieren\grqq\  - wenn auch nicht unbedingt in unserem Kopf. Weiter hieße das, dass \glqq etwas\grqq\ existiert, dieses \glqq etwas\grqq , kann jedoch auch rein informationell sein, im Sinne von nicht-substanziell, wie es beispielsweise auch der Begriff \glqq digital\grqq\ charakterisiert.
%TODO verbessern; zur Not raus

% Ende:
Wenn man weiter prüft, muss man eingestehen, dass der Skeptizismus in \textbf{der} Hinsicht von Nutzen ist, als dass er einen mit dem Mittel ausstattet, aktiv vom täglichen praktischen Dogmatismus zurückzukehren und alles zu hinterfragen.
%\subsection{Schluss}
\nolinenumbers
%\printbibliography %TODO quickbuild: bib
\end{document}



%TODO
 - Schaubild 
 - geschichtlicher Hintergrund
 - evtl. Nihilismus
 - Popper
 - Kritizismus weiter ausbauen

\glqq Was ist jede Erkenntnis, wenn sie nicht auf Logik basiert;\\
was bringt jede Logik, wenn das Leben nicht nach ihr funktioniert.\grqq\


Übergangsliste:
Antike -(Akademie)-> MittelA. -(Augustinus \glqq cogito ergo sum\grqq\ )-> Rat.+Emp. -(Kant)-> Moderne


---------------
490 BC -  420 BC Protagoras (-> Sophisten)
470/469 BC - 399 BC Sokrates (-> Sophisten)
360 BC - 270 BC Pyrrho (?)
266 BC - 90 BC  Arcesilaus - akademie


Socrates had said, \glqq This alone I know, that I know nothing.\grqq\ But Arcesilaus went farther and denied the possibility of even the Socratic minimum of certainty: \glqq I cannot know even whether I know or not.\grqq\
(One or more of the preceding sentences incorporates text from a publication now in the public domain: Chisholm, Hugh, ed. (1911). \glqq Academy, Greek\grqq. Encyclopædia Britannica (11th ed.). Cambridge University Press.
\text{http://en.wikisource.org/wiki/1911_Encyclop%C3%A6dia_Britannica/Academy,_Greek}
)


Anmerkung: evtl. Zeitdaten
