\documentclass[12pt]{beamer}
\usetheme[hideothersubsections]{Berkeley}
\usepackage[utf8]{inputenc}
\usepackage[german]{babel}
\usepackage[T1]{fontenc}
\usepackage{graphicx} %TODO Logo!
\graphicspath{{images/}}
\usepackage{csquotes}

\author[L. König]{Leonard König}
\title[Skeptizismus]{Skeptizismus - Flucht der Philosophie vor der Frage nach der Erkenntnis?}
%\setbeamercovered{transparent} 
%\setbeamertemplate{navigation symbols}{} 
\logo{\includegraphics[scale=0.4]{logo.png}} 
\institute{Herder-Gymnasium} 
%\date{} 
\subject{Skeptizismus} 
\keywords{Philosophie, Erkenntnistheorie, Metaphysik, Skeptizismus}

% personalpronomen? Ja, Nein?

%\setlength{\parskip}{1em}
\usepackage[
	natbib=true,
    backend=biber,
    style=numeric,
    citestyle=numeric,
    sorting=nty,
    sortlocale=de_DE,
]{biblatex}
\addbibresource{text.bib}

\begin{document}
\begin{frame}
\titlepage
\end{frame}

\begin{frame}
\tableofcontents
\end{frame}


\section{Hinführung zum Thema: Sokrates}
\begin{frame}{\glqq Ich weiß, dass ich nichts weiß\grqq}
\begin{center}
\glqq Ich weiß, dass ich nichts weiß\grqq\\
\ \\
Sokrates: Skeptiker oder methodischer Zweifler?
\end{center}
\end{frame}

\subsection{Definitionen}
\subsubsection{Skeptizismus}
\begin{frame}{Def.: Skeptizismus}
Duden:\\
\glqq Im weiteren Sinn philosophische Positionen, die Wahrheitsansprüchen gegenüber Verzicht und Zurückhaltung üben und sorgfältige kritische Prüfung verlangen.\\
Im strengeren Sinne Richtungen, die die Beweisbarkeit von Wahrheit (entweder prinzipiell oder partiell) in Zweifel ziehen.\grqq\\
\ \\
Beispiel für einen modernen Skeptiker: David Hume
\end{frame}		

\subsubsection{Agnostizismus}
\begin{frame}{Def.: Agnostizismus}
Agnostizismus:
\begin{itemize}
\item geprägt von Thomas Henry Huxley
\item bezog sich vor Allem auf den Glauben
\item[$\Rightarrow$] Unabhängigkeit der Wissenschaft von Religion herstellen
\item Herkunft: $\grave{\alpha}\gamma\nu\bar{\omega}\sigma\iota\varsigma$: \glqq ohne Wissen\grqq\
\item[$\Rightarrow$] vereinzelt synonym zu Skeptizismus verwendet
\end{itemize}
\end{frame}

\subsubsection{Solipsismus}
\begin{frame}{Def.: Solipsismus}
Solipsismus:
\begin{itemize}
\item skeptische Haltung ggü. allem außer der eigenen Selbst
\item nur die eigene Existenz gesichert
\item oder: Bedeutung der Wahrnehmung abhängig vom Zustand des (denkenden) Ichs %\cite{iep_solipsis} %TODO
\end{itemize}
\end{frame}

\section{Skeptizismus im historischen Kontext}
\subsection{Antike - die Sophisten} %TODO Ordnen, historisch u. nach Position
\begin{frame}{Antike - die Sophisten}
Sophisten waren Wanderlehrer, sie unterrichteten
\begin{itemize}
\item Diskussion
\item Politik
\item Geschichte
\item \emph{Philosophie \& Erkenntnistheorie}
\end{itemize}
Jedoch entstanden, zeitbedingt, viele unterschiedliche Schulen
%TODO Zeitbedingt? Rechtschreibung!
\end{frame}

\begin{frame}{Protagoras und Gorgias}
Protagoras: Mensch als Maß aller Dinge\\% \cite{enc_brit_sophist} %TODO
\ \\
Gorgias: dogmatischer Skeptizismus:
\begin{itemize}
\item Nichts existiert;
\item Wenn etwas existiert, dann kann man nichts darüber wissen; und
\item Wenn man etwas darüber wissen kann, dann kann dieses nicht mitgeteilt werden.% \cite{iep_on-the-nonex} %TODO
\end{itemize}
\end{frame}

\begin{frame}{Radikaler vs. akademischer Skeptizismus}
\textbf{Akademischer Skeptizismus:}\\
$\rightarrow$ keinerlei Wissen möglich\\
\textbf{Radikaler Skeptizismus:}\\
$\rightarrow$ kritische Ansichten gegenüber \emph{jeder} Aussage:\\
\glqq unsere Sinneserfahrungen und Ansichten sind weder wahr, noch falsch\grqq\ %\cite{greek_stough} %TODO
% Der Skeptizismus nach Pyrrho geht nicht so dogmatisch vor, sondern behauptet, man könne über die Aussage selbst, also, dass kein Wissen möglich ist, keine wirkliche Aussage treffen kann (radikaler Skeptizismus). % %TODO http://en.wikipedia.org/wiki/Philosophical_skepticism
\end{frame}

%Der siebte Leiter der antiken Akademie, Arkesilaos, begründete den akademischen Skeptizismus. Er schloss sich dabei seinen Vorgängern an, dass es keine Wahrheit gäbe, weil zu beliebigen (philosophischen) Aussagen gleichberechtigte Gegenargumente oder Positionen zu finden seien. %TODO contradiction!!!
\subsection{Mittelalter - akademischer Skeptizismus versus Kirche}
%TODO Bis hier: 5min
\begin{frame}{Augustinus}
Augustinus:
\begin{itemize}
\item von Ciceros akademischen Skeptizismsus geprägt
\item Römerbriefen: Religiosität und Kritik am Skeptizismus
\item trotzdem Platonische Auslegung der Bibel
\end{itemize}
\glqq Wird jemand darüber zweifeln, dass er lebt, sich erinnert, Einsichten hat, will, denkt, weiß und urteilt? [\ldots] Mag einer auch sonst zweifeln, über was er will, über diese Zweifel selbst kann er nicht zweifeln\grqq\ %\cite{de_trini_x} %TODO
\end{frame}
%TODO Zitat hervorheben, evtl. Manichäismus raus; dafuer mehr Popper
\subsection{Skeptizismus von Descartes bis Kant}
\begin{frame}{Descartes}
Descartes:
\begin{itemize}
\item Übernahm den methodischen Zweifel und  \glqq ich zweifle, also bin ich\grqq
\item Versuch durch reine Logik zu beweisbarer Erkenntnis zu gelangen
\item Dadurch, dass man daran zweifelt, dass man zweifelt, beweise man das eigene Zweifeln
\end{itemize}
\emph{Doch: Weiß man, dass man wirklich zweifelt?}
\end{frame}

\begin{frame}{David Hume}
David Hume:
\begin{itemize}
\item Empiriker: Erkenntnisgewinn nur durch Erfahrung
\item[$\Rightarrow$] es liegt nahe, dass unsere Erfahrung getäuscht ist
\item[$\Rightarrow$] evtl. kein \glqq echter\grqq\ Erkenntnisgewinn möglich
\end{itemize}
$\Rightarrow$ Was ist überhaupt \glqq Erkenntnis\grqq ?
\end{frame}

\begin{frame}{Immanuel Kant}
Herangehensweise abseits der Logik:\\
Zu einer  \glqq gereiften [\ldots] Urteilskraft\grqq\ gehöre es auch, dass diese \glqq feste und ihrer Allgemeinheit nach bewährte Maximen zum Grunde hat\grqq\ %\cite{kritik} %TODO
\begin{itemize}
\item[$\Rightarrow$] wichtig gegen Dogmatismus und zur Reflexion des Lebens
\item[$\Rightarrow$] jedoch nicht um die eigene Lethargie zu rechtfertigen, nach dem Motto:\\
\glqq Ich kann ja sowieso nichts wissen\grqq
\end{itemize}
\end{frame}

\subsection{Moderner Skeptizismus}
%TODO 10min
%TODO Kritizismus erlaeutern
\begin{frame}{Ethik: Der Nihilismus}
Ethik:\\
\begin{itemize}
\item logisch folgerichtige \glqq Beweisführung\grqq\ für einen Nihilismus
\item jedoch keine Rechtfertigung auf moralischer Basis?
\end{itemize}
Nietzsche:\\
\begin{itemize}
\item Verlust der Metaphysik durch unendlichen Zweifel
\item[$\Rightarrow$] kein Fundament der Moral mehr gegeben
\item[$\Rightarrow$] Nihilismus ist eine Entwertung der obersten Werte der Moral
\item \glqq Umwertung aller Werte\grqq als neue Basis für die Ethik
\end{itemize}
%TODO Verlust der Metaphysik: Nachrecherchieren! Axiome?
\end{frame}

\begin{frame}{Von Nietzsche zu Nelson}
Nelson: \glqq Die Unmöglichkeit der Erkenntnistheorie\grqq
%TODO fuellen
\end{frame}
%TODO Ueberleitung

\begin{frame}{Popper: Kritischer Realismus}
\begin{itemize}
\item Wissenschaftstheorie: Angewandter Skeptizismus:
\item[$\Rightarrow$] Kritik an Induktion etc.
\item andererseits soll über Erfahrung wissenschaftlich gesicherte Erkenntnis gewinnbar sein
\end{itemize}
\glqq Kritischer Realismus\grqq
%TODO %Diese Position nennt man kritischen Realismus.% Rationalismus? sicher? uebepruefen!
\end{frame}

\section{Reflexion - Skeptizismus als Lebenseinstellung?}
\begin{frame}
\begin{itemize}
\item Sicht Skeptizismus im Kontext der Leitfrage?\\
verschiedene Haltungen des Skeptizismus differenziert zu betrachten?
\item logisch konsistent?
\item epistemologischer Sinn?\\
sinnvoll in abstrakterem Kontext? %konkreter
% ende der reflexion: aber eine frage habe ich vergessen:
% der skeptizismus ist erst eine flucht, wenn man ihn benutzt, sonst bleibt er eine moeglichkeit
% Und da "die Philosophie" nicht "fluechten" kann, ist es eben nur letzteres, maximal
\end{itemize}
\end{frame}

\begin{frame}{Antworten}
Vor allem erkenntnistheoretische und metaphysische Aspekte relevant:\\
$\Rightarrow$ Radikaler Skeptizismus und daraus abgeleitete Positionen\\

Durch Begrenzen auf die Logik, eine in sich sehr konsistente Bearbeitung eines abstrakten Themas\\
Mag \glqq einfach\grqq wirken, dh. jedoch noch keine \glqq Flucht\grqq .
%TODO verstichpunkten
\end{frame}

\begin{frame}{Skeptizismus heißt Zweifeln}
Skeptizismus heißt Zweifeln.\\
Zwei Möglichkeiten des Zweifels:
\begin{itemize}
\item[1.] Um zum Schluss der Allanzweifelbarkeit zu kommen\\
\item[$\rightarrow$] Lethargie, eine logisch \glqq beweisbare\grqq\ Position
%TODO Allanzweifelbarkeit? Sortieren (so.)
\item[2.] Um zu versuchen, die eigene Erkenntnis zu erweitern
\item[$\rightarrow$] Zwangsläufig ebenfalls kein Erkenntnisgewinn
\end{itemize}
\end{frame}

\begin{frame}{Das Ende?}
\emph{Doch} eine Flucht?\\
\ \\
\begin{center}
\emph{Dies wäre der Fall, wenn eine Erkenntnistheorie nur an ihrem Erkenntnisgewinn \textbf{an sich} gemessen werden würde!}
\end{center}
\end{frame}

\begin{frame}{Der Meta-Sinn}
Meta-Sinn:
\begin{itemize}
\item Suche nach Abstraktion, und purer Logik
\item Keine Sicherheit, Bodenlosigkeit auch als Freiheit interpretierbar
\item[$\Rightarrow$] Ziel ist nicht: Erkenntnisgewinn
\end{itemize}
\end{frame}


\begin{frame}{Das \glqq Universum\grqq\ des Skeptizismus}
Vor.: Wir \glqq glauben\grqq wir denken. wissen aber nicht, was wir eigentlich tun, falls nicht \glqq glauben\grqq \\
Fall \RN{1}: Wir denken. Ergo, existieren \glqq wir\grqq . \\
Fall \RN{2}: Wir denken nicht. Ergo \glqq existiert\grqq\ irgendetwas, was uns glauben macht, zu denken\\
\ \\
Beispielsweise: Ein Programm, Algorithmus, \ldots - eine \glqq Steuerung\grqq\ von \glqq außen\grqq
\end{frame}

\begin{frame}
\begin{center}
Es existiert \glqq etwas\grqq .\\
Darin liegen die \glqq Informationen\grqq\ unseres \glqq Denkens\grqq
\end{center}
\end{frame}

\begin{frame}{Ende}
Ziel des Skeptizismus ist gleich der Intention des Skeptikers\\
$\rightarrow$ Skeptizismus bloße Feststellung der Erkenntnislosigkeit\\
\begin{center}
Auch ein Mittel der Reflexion, der Rückkehr vom täglichen Dogmatismus
\end{center}
\end{frame}
%%\printbibliography %TODO quickbuild: bib}
%
%
%

\end{document}