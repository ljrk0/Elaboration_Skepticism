\documentclass[12pt]{beamer}
\usetheme[hideothersubsections]{Berkeley}
\usepackage[utf8]{inputenc}
\usepackage[polutonikogreek,ngerman]{babel}
\usepackage[T1]{fontenc}
\usepackage{graphicx}
\graphicspath{{images/}}
\usepackage{csquotes}
\usepackage{textgreek}

\author{Leonard König}
\title[Skeptizismus]{Skeptizismus – Flucht der Philosophie vor der Frage nach der Erkenntnis?}
%\setbeamercovered{transparent} 
%\setbeamertemplate{navigation symbols}{} 
\logo{\includegraphics[scale=0.4]{logo.png}} 
\institute{Herder-Gymnasium} 
%\date{} 
\subject{Skeptizismus} 
\keywords{Philosophie, Erkenntnistheorie, Metaphysik, Skeptizismus}

% personalpronomen? Ja, Nein?

%\setlength{\parskip}{1em}
\usepackage[
	natbib=true,
    backend=biber,
    style=numeric,
    citestyle=numeric,
    sorting=nty,
    sortlocale=de_DE,
]{biblatex}
\addbibresource{text.bib}

\makeatletter
\newcommand*{\rom}[1]{\expandafter\@slowromancap\romannumeral #1@}
\makeatother

\begin{document}
\begin{frame}{Skeptizismus}{Flucht der Philosophie vor der Frage nach der Erkenntnis?}
\titlepage
\end{frame}

\begin{frame}{Gliederung}
\tableofcontents
\end{frame}


\section{Hinführung zum Thema: Sokrates}
\begin{frame}{\glqq Ich weiß, dass ich nichts weiß\grqq}
\begin{center}
\glqq Ich weiß, dass ich nichts weiß\grqq\\
\ \\
Sokrates: Skeptiker oder methodischer Zweifler?
\end{center}
\end{frame}

\subsection{Definitionen}
\subsubsection{Skeptizismus}
\begin{frame}{Def.: Skeptizismus}
Duden:\\
\glqq Im weiteren Sinn philosophische Positionen, die Wahrheitsansprüchen gegenüber Verzicht und Zurückhaltung üben und sorgfältige kritische Prüfung verlangen.\\
Im strengeren Sinne Richtungen, die die Beweisbarkeit von Wahrheit (entweder prinzipiell oder partiell) in Zweifel ziehen.\grqq\\
\ \\
Beispiel für einen modernen Skeptiker: David Hume
\end{frame}		

\subsubsection{Agnostizismus}
\begin{frame}{Def.: Agnostizismus}
Agnostizismus:
\begin{itemize}
\item geprägt von Thomas Henry Huxley
\item bezog sich vor allem auf den Glauben
\item[$\Rightarrow$] Unabhängigkeit der Wissenschaft von Religion herstellen
\item Herkunft: \\
{\selectlanguage{polutonikogreek}\`{\textalpha}γν\={\textomega}σις}: \glqq ohne Wissen\grqq\
\item[$\Rightarrow$] vereinzelt synonym zu Skeptizismus verwendet
\end{itemize}
\end{frame}

\subsubsection{Solipsismus}
\begin{frame}{Def.: Solipsismus}
Solipsismus:
\begin{itemize}
\item Skepsis bezüglich allem außer sich selbst
\item nur die eigene Existenz gesichert
\item oder: Bedeutung der Wahrnehmung abhängig vom Zustand des (denkenden) Ichs
\end{itemize}
\end{frame}

\section{Skeptizismus im historischen Kontext}
\subsection{Antike – die Sophisten}
\begin{frame}{Antike – die Sophisten}
Sophisten waren Wanderlehrer, sie unterrichteten
\begin{itemize}
\item Diskussion
\item Politik
\item Geschichte
\item \emph{Philosophie \& Erkenntnistheorie}
\end{itemize}
Jedoch entstanden – zeitbedingt – viele unterschiedliche Schulen
\end{frame}

\begin{frame}{dogmatischer Skeptizismus}
Gorgias: dogmatischer Skeptizismus:
\begin{itemize}
\item Nichts existiert;
\item Wenn etwas existiert, dann kann man nichts darüber wissen; und
\item Wenn man etwas darüber wissen kann, dann kann dieses nicht mitgeteilt werden.
\end{itemize}
\end{frame}

\begin{frame}{Radikaler und akademischer Skeptizismus}
\textbf{Radikaler Skeptizismus:}\\
$\rightarrow$ kritische Ansichten gegenüber \emph{jeder} Aussage:\\
\glqq Sinneserfahrungen und Ansichten  weder wahr noch falsch\grqq\\
\textbf{Akademischer Skeptizismus:}\\
\begin{itemize}
\item Eingeläutet durch Arkesilaos in der Akademie
\item[$\rightarrow$] keinerlei Wissen möglich
\end{itemize}
\end{frame}

\subsection{Mittelalter – Augustinus}
%TIME Bis hier: 5min
\begin{frame}{Augustinus}
Augustinus:
\begin{itemize}
\item von Ciceros akademischen Skeptizismus geprägt
\item \glqq Römerbriefe\grqq : Religiosität und Kritik am Skeptizismus
\item trotzdem Platonische Auslegung der Bibel
\end{itemize}
\glqq Wird jemand darüber zweifeln, dass er lebt, sich erinnert, Einsichten hat, will, denkt, weiß und urteilt? [\ldots] Mag einer auch sonst zweifeln, über was er will, über diese Zweifel selbst kann er nicht zweifeln\grqq
\end{frame}
%TODO Zitat hervorheben, evtl. Manichäismus raus; dafuer mehr Popper
\subsection{Skeptizismus von Descartes bis Kant}
\begin{frame}{Descartes}
Descartes:
\begin{itemize}
\item Übernahm  methodischen Zweifel und  \glqq ich zweifle, also bin ich\grqq
\item Versuch durch reine Logik zu beweisbarer Erkenntnis zu gelangen
\item Dadurch, dass man daran zweifelt, dass man zweifelt, beweise man das eigene Zweifeln
\end{itemize}
\emph{Doch: Weiß man, dass man wirklich zweifelt?}
\end{frame}

\begin{frame}{David Hume}
David Hume:
\begin{itemize}
\item Empiriker: Erkenntnisgewinn nur durch Erfahrung
\item[$\Rightarrow$] es liegt nahe, dass unsere Erfahrung getäuscht ist
\item[$\Rightarrow$] kein \glqq echter\grqq\ Erkenntnisgewinn möglich?
\end{itemize}
\ \\
\ \\
$\Rightarrow$ Was ist überhaupt \glqq Erkenntnis\grqq ?
\end{frame}

\begin{frame}{Immanuel Kant}
Immanuel Kant:\\
Herangehensweise abseits der Logik:\\
Zu einer  \glqq gereiften [\ldots] Urteilskraft\grqq\ gehöre es auch, dass diese \glqq feste und ihrer Allgemeinheit nach bewährte Maximen zum Grunde hat\grqq
\begin{itemize}
\item[$\Rightarrow$] wichtig gegen Dogmatismus und zur Reflexion des Lebens
\item[$\Rightarrow$] jedoch nicht um die eigene Lethargie zu rechtfertigen nach dem Motto:\\
\glqq Ich kann ja sowieso nichts wissen\grqq
\end{itemize}
\end{frame}

\subsection{Moderner Skeptizismus}
%TIME 10min
%TODO Kritizismus erlaeutern
\begin{frame}{Ethik: Der Nihilismus – Nietzsche}
Ethisches Terrain:\\
\begin{itemize}
\item Begründung, jedoch keine Rechtfertigung?
\item[$\rightarrow$] jedoch keine Rechtfertigung auf moralischer Basis?
\end{itemize}
Nietzsche:\\
\begin{itemize}
\item Verlust der Metaphysik durch unendlichen Zweifel
\item[$\Rightarrow$] kein Fundament der Moral mehr gegeben
\item[$\Rightarrow$] Nihilismus ist eine Entwertung der obersten Werte der Moral
\item \glqq Umwertung aller Werte\grqq\ als neue Basis für die Ethik
\end{itemize}
%TODO Verlust der Metaphysik: Nachrecherchieren! Axiome? weiter verbessern
\end{frame}


\begin{frame}{Popper: Kritischer Realismus}
Popper:\\
\begin{itemize}
\item Wissenschaftstheorie: Angewandter Skeptizismus:
\item[$\Rightarrow$] Kritik an Induktion und ähnlichen empirischen Forschungen
\item andererseits soll über Erfahrung wissenschaftlich gesicherte Erkenntnis gewinnbar sein
\end{itemize}
$\rightarrow$ \glqq Kritischer Realismus\grqq
%TODO %Diese Position nennt man kritischen Realismus.% Rationalismus? sicher? uebepruefen!
\end{frame}

\section{Reflexion – Skeptizismus als Lebenseinstellung?}
\begin{frame}{Fragen}
Fragen:
\begin{itemize}
\item Sicht auf den Skeptizismus im Kontext der Leitfrage?\\
verschiedene Haltungen des Skeptizismus differenziert zu betrachten?
\item logisch konsistent?
\item epistemologischer Sinn?\\
sinnvoll in abstrakterem Kontext? %konkreter
% ende der reflexion: aber eine frage habe ich vergessen:
% der skeptizismus ist erst eine flucht, wenn man ihn benutzt, sonst bleibt er eine moeglichkeit
% Und da "die Philosophie" nicht "fluechten" kann, ist es eben nur letzteres, maximal
\end{itemize}
\end{frame}

\begin{frame}{Antworten}
Antworten:\\
Beschränkung auf Erkenntnistheorie und Metaphysik:\\
$\Rightarrow$ Radikaler Skeptizismus\\
\ \\
\glqq triviale\grqq , logische Bearbeitung eines komplexen Themas\\
$\rightarrow$ trotzdem keine \glqq Flucht\grqq
\end{frame}

\begin{frame}{Skeptizismus bedeutet Zweifel}
\textbf{Skeptizismus bedeutet Zweifel.}\\
Zwei Möglichkeiten des Zweifels:
\begin{itemize}
\item[1.] \textbf{Destruktiv:} Ziel: Schluss der Allanzweifelbarkeit
\item[$\rightarrow$] Lethargie, eine logisch \glqq beweisbare\grqq\ Position
\item[2.] \textbf{Konstruktiv:} Ziel: Erkenntnis erweitern
\item[$\rightarrow$] Zwangsläufig ebenfalls kein Erkenntnisgewinn
\end{itemize}
\end{frame}

\begin{frame}{Das Ende?}
\emph{Doch} eine Flucht?\\
\ \\
\begin{center}
\emph{Dies wäre der Fall, wenn eine Erkenntnistheorie nur an ihrem Erkenntnisgewinn \textbf{an sich} gemessen werden würde!}
\end{center}
\end{frame}

\begin{frame}{Der Meta-Sinn}
Meta-Sinn:
\begin{itemize}
\item Suche nach Abstraktion, und purer Logik
\item Keine Sicherheit, Bodenlosigkeit auch als Freiheit interpretierbar
\item[$\Rightarrow$] Ziel ist nicht: Erkenntnisgewinn
\end{itemize}
\end{frame}


\begin{frame}{Das \glqq Universum\grqq\ des Skeptizismus}
Vor.: Wir \glqq glauben\grqq\ wir denken. wissen aber nicht, was wir eigentlich tun, falls nicht \glqq glauben\grqq \\
\ \\
Fall \rom{1}: Wir denken. Ergo, existieren \glqq wir\grqq . \\
Fall \rom{2}: Wir denken nicht. Ergo \glqq existiert\grqq\ irgendetwas, was uns glauben macht, zu denken\\
\ \\
Beispielsweise: Ein Programm, Algorithmus, \ldots - eine \glqq Steuerung\grqq\ von \glqq außen\grqq
\end{frame}

\begin{frame}{Existenz von Informationen}
\begin{center}
Es existiert \glqq etwas\grqq .\\
Darin liegen die \glqq Informationen\grqq\ unseres \glqq Denkens\grqq
\end{center}
\end{frame}

\begin{frame}{Ende}
Ziel des Skeptizismus folgt aus der Intention des Skeptikers\\
$\rightarrow$ Skeptizismus bloße Feststellung der Erkenntnislosigkeit\\
\begin{center}
Außerdem ein Mittel der Reflexion, der Abkehr vom täglichen Dogmatismus
\end{center}
\end{frame}

\begin{frame}{Danke!}
\begin{center}
{\Large Vielen Dank für Ihre Aufmerksamkeit!}
\end{center}
\end{frame}

\nocite{*}

\begin{frame}[t, allowframebreaks]{Quellen}
\printbibliography
\end{frame}

%
%
%

\end{document}
